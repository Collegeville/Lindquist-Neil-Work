\documentclass[acmsmall]{acmart}
\bibliographystyle{ACM-Reference-Format}

\usepackage{array}
\usepackage{tikz}
\usetikzlibrary{positioning}

\title{JuliaPetra: An Implementation of the Petra Object Model in Julia}

\author{Neil Lindquist}
\email{nslindquist@csbsju.edu}

\acmJournal{TOMS}


\newcommand{\snippet}[1]{\texttt{\detokenize{#1}}}

\begin{document}

%REVIEW "Distributed algorithms → Self-organization" concept displayed as "self-organization"
%TODO make sure the concepts represent what I think they do
\begin{CCSXML}
<ccs2012>
	<concept>
		<concept_id>10002950.10003714.10003715.10003719</concept_id>
		<concept_desc>Mathematics of computing~Computations on matrices</concept_desc>
		<concept_significance>500</concept_significance>
	</concept>
	<concept>
		<concept_id>10002950.10003705</concept_id>
		<concept_desc>Mathematics of computing~Mathematical software</concept_desc>
		<concept_significance>100</concept_significance>
	</concept>
	<concept>
		<concept_id>10010147.10010148.10010149.10010158</concept_id>
		<concept_desc>Computing methodologies~Linear algebra algorithms</concept_desc>
		<concept_significance>500</concept_significance>
	</concept>
	<concept>
		<concept_id>10010147.10010919.10010172.10003824</concept_id>
		<concept_desc>Computing methodologies~Self-organization</concept_desc>
		<concept_significance>100</concept_significance>
	</concept>
</ccs2012>
\end{CCSXML}

\ccsdesc[500]{Mathematics of computing~Computations on matrices}
\ccsdesc[100]{Mathematics of computing~Mathematical software}
\ccsdesc[500]{Computing methodologies~Linear algebra algorithms}
\ccsdesc[100]{Computing methodologies~Self-organization}

\begin{abstract}
JuliaPetra provides linear algebra data structures that are commonly used in large scale, linear solver algorithms.
The data structures in JuliaPetra focus on vectors and sparse matrices, in both serial and
distributed, parallel environments.
The library is written in Julia, a high level programming language with comparable performance to C or Fortran.
JuliaPetra performs as fast as Epetra, an equivalent C++ library, and faster than DistributedArrays.jl, a Julia
library for distributed computations on arrays.
\end{abstract}

\maketitle

\section{Introduction}

JuliaPetra is an implementation of the Petra object model in Julia.
The Petra object model is the design used in Trilinos for objects commonly used by linear solver algorithms
\cite{Heroux:2005:Trilinos}.
Petra libraries provide parallel, distributed matrices, vectors and graphs for other packages to use.
By providing a set of interfaces for basic linear algebra structures, Petra libraries provide a way
to develop packages for these types of distributed algorithms that are able to interact and build on each other.
The Petra object model has previously been implemented in both C++ and Java.
%REVIEW should this list of previous implemenetations be in the paper?

%REVIEW does there need to be a section on Julia?
	%There isn't really an equivalent in papers for Fortrain/C programs
	%Julia is less estabilished for hpc though
Julia is a programing language that uses just in time compiling and powerful type inferencing
to obtain the speed of a statically compiled programing language with the productivity of an
interpreted programing language \cite{Bezanson:2017:FreshApproach}.
By providing speed comparable to that of Fortran or C, Julia is a candidate for doing large scale,
high performance computations.
The Celeste project is an example of large scale computations in Julia,
using 8,192 cores of the Cori Supercomputer
at Lawrence Berkeley National Laboratory \cite{Bezanson:2017:FreshApproach}.

%REVIEW this is a strange paragraph.
Like other implementations of the Petra object model,
JuliaPetra provides the ability to do basic linear algebra computations.
These computations include dot products, vector norms and sparse matrix - vector multiplications.
This document provides, first, an overview of the types and interfaces of JuliaPetra,
second, a summary of optimizations used to implement JuliaPetra,
and finally, comparisons with two similar libraries, Epetra and DistributedArrays.jl.

\section{Design of JuliaPetra}

The design of JuliaPetra follows closely with that of the C++ implementations of the
Petra object model, Epetra and Tpetra.
Like Epetra and Tpetra, JuliaPetra uses MPI and Single-Program-Multiple-Data as its base parallel model.
The API is split into two main layers of abstraction, the communication layer and the data structures layer.

The first level of abstraction handles the communication details and how the problem is
distributed across processes.
The \snippet{Comm} and \snippet{Distributor} types provide a low level interface to support
different communication methods.
Because all interprocess communication is done through these objects,
new communication systems can be implemented without affected the objects built on top of the communication layer.
The \snippet{BlockMap}, \snippet{Directory}, \snippet{Import} and \snippet{Export}
types handle the distribution of the problem across the processors.
\snippet{BlockMap} and \snippet{Directory} manage which processor the various parts of
the problem are located.
\snippet{Import} and \snippet{Export} contain the logic behind redistributing
the problem among the processes.
The \snippet{SrcDistObject} and \snippet{DistObject} interfaces provide a connection between
the redistributing logic and the data structures themselves.
The communication layer provides an abstraction on which the distributed data structures can build.

The linear algebra objects are built on top of the abstractions provided by the communication layer.
JuliaPetra has two main linear algebra types.
The first type is the abstract type \snippet{MultiVector} which holds one or more vectors.
This type is implemented by \snippet{DenseMultiVector} which stores dense vectors.
The second type is the \snippet{Operator} interface which is implemented by types that provide a
\(y \gets \alpha\cdot A(x) + \beta\cdot y\) operation, where \(\alpha\) and \(\beta\) are scalars,
\(x\) and \(y\) are vectors and \(A\) is the operator.
The abstract type \snippet{RowMatrix}, which stores sparse matrices accessed by rows,
supports this interface by multiplying the matrix on the left of \(x\).
\snippet{RowGraph} is an additionally type used to represent the sparsity pattern of a \snippet{RowMatrix}.
Both \snippet{RowMatrix} and \snippet{RowGraph} have concrete implementations based on
compressed sparse row format, \snippet{CSRMatrix} and \snippet{CSRGraph} respectively.
Julia's \snippet{AbstractArray} is subtyped by both the \snippet{MultiVector}
and \snippet{RowMatrix} types to allow existing Julia code to interact with data in JuliaPetra objects.

\begin{figure}
	%name, type, location
	\newcommand{\typeNode} [3]{\node[#2] (#1) [#3] {\snippet{#1}};}
	\newcommand{\explicitInheritance}[2]{\draw[->] (#1.south) -- (#2);}
	\newcommand{\implicitInheritance}[2]{\draw[dashed,->] (#1.south) -- (#2);}
	\begin{tikzpicture}[
	ImplicitInterface/.style={rectangle, draw=black, dashed, minimum size=5mm},
	ExplicitInterface/.style={rectangle, draw=black, minimum size=5mm},
	ConcreteType/.style={rectangle, draw=black, very thick, minimum size=5mm}
	]

	\typeNode{Comm}{ExplicitInterface}{}

	\typeNode{MPIComm}{ConcreteType}{below=.75cm of Comm}
	\explicitInheritance{Comm}{MPIComm.north}

	\typeNode{SerialComm}{ConcreteType}{right=.25cm of MPIComm}
	\explicitInheritance{Comm}{SerialComm.north}

	\typeNode{LocalComm}{ConcreteType}{left=.25cm of MPIComm}
	\explicitInheritance{Comm}{LocalComm.north}

	\typeNode{Distributor}{ExplicitInterface}{left=5.1cm of Comm}

	\typeNode{SerialDistributor}{ConcreteType}{below right=.75cm and -1cm of Distributor}
	\explicitInheritance{Distributor}{SerialDistributor.north}

	\typeNode{MPIDistributor}{ConcreteType}{left=.25cm of SerialDistributor}
	\explicitInheritance{Distributor}{MPIDistributor.north}



	\typeNode{Directory}{ExplicitInterface}{below right=.75cm and -1.25cm of MPIDistributor}

	\typeNode{BasicDirectory}{ConcreteType}{below=.75cm of Directory}
	\explicitInheritance{Directory}{BasicDirectory.north}

	\typeNode{BlockMap}{ConcreteType}{right=of Directory}

	\typeNode{Import}{ConcreteType}{right=of BlockMap}

	\typeNode{Export}{ConcreteType}{right=of Import}



	\typeNode{SrcDistObject}{ImplicitInterface}{below left=.75cm and -.9cm of Export}

	\typeNode{DistObject}{ImplicitInterface}{below=.75cm of SrcDistObject}
	\implicitInheritance{SrcDistObject}{DistObject.north}

	\typeNode{Operator}{ImplicitInterface}{left=of DistObject}

	\typeNode{AbstractArray}{ExplicitInterface}{left=of Operator}

	\typeNode{MultiVector}{ExplicitInterface}{below=of AbstractArray}
	\implicitInheritance{DistObject}{MultiVector.north east}
	\explicitInheritance{AbstractArray}{MultiVector.north}
	
	\typeNode{DenseMultiVector}{ConcreteType}{below=.75cm of MultiVector}
	\explicitInheritance{MultiVector}{DenseMultiVector.north}

	\typeNode{RowMatrix}{ExplicitInterface}{below=of Operator}
	\implicitInheritance{DistObject}{RowMatrix.north east}
	\implicitInheritance{Operator}{RowMatrix.north}
	\explicitInheritance{AbstractArray}{RowMatrix.north west}

	\typeNode{RowGraph}{ExplicitInterface}{below=of DistObject}
	\implicitInheritance{DistObject}{RowGraph.north}

	\typeNode{CSRMatrix}{ConcreteType}{below=.75cm of RowMatrix}
	\explicitInheritance{RowMatrix}{CSRMatrix.north}

	\typeNode{CSRGraph}{ConcreteType}{below=.75cm of RowGraph}
	\explicitInheritance{RowGraph}{CSRGraph.north}



	\node[ExplicitInterface] (ExplicitKey) [below=of CSRMatrix] {Abstract Type};
	\node[ImplicitInterface] (ImplicitKey) [left=of ExplicitKey] {Implicit Type};
	\node[ConcreteType]      (ConcreteKey) [right=of ExplicitKey] {Concrete Type};

	\end{tikzpicture}
	\caption{Type hierarchy.}
	\label{fig:type-hierarchy}
\end{figure}

The type hierarchy in JuliaPetra is limited by the fact that types in Julia are restricted to a single supertype.
Interacting with existing code as a 2-dimensional array requires being a subtype of \snippet{AbstractArray}.
So, the other interfaces for the data structures are not explicit types,
but simply contracts promised in the documentation.
These implicit interfaces include \snippet{Operator}, \snippet{SrcDistObject}
and \snippet{DistObject}.
So, any methods that use one of those types accepts an object of any type
and the documentation specifies the methods that must be implemented.
Figure~\ref{fig:type-hierarchy} shows the full type hierarchy.

\section{Optimizations of JuliaPetra}
%This is to give insight on best practices for people who don't look at the general Julia discussions
%Talk about the "story" of improving the code and how optimizations improved performance.

There were alot of optimizations used to optimize JuliaPetra,
many of which are standard adivice for optimizing Julia, such as type stability,
or for optimization in general, such as accessing arrays in memory order.
The only major unusual optimization was using \snippet{Ptr} objects to pass views of arrays
in an effort to reduce allocation and garbage collection.

The main improvements to performance in JuliaPetra came from improving type stability.
Type stability is where the compiler can inference the types of all values.
This allows the compiler is able to fully inline function calls, especially arithmetic and array operations.
Because Julia is designed to support both slower, type unstable code and faster, type stable code,
unlike C++ which is restricted to type stable code, some effort needs to be put into verifying
type stability of performance critical code \cite{Bezanson:2017:FreshApproach}.
%REVIEW mention that this is standard Julia advice?
Removing type instabilities from performance critical sections of JuliaPetra improved performance
by a few orders of magnitude.
The \snippet{code_warntype} function was an important tool in finding type instabilities.
Due to \snippet{code_warntype} requiring manual inspection, the TypeStability.jl library was created
based on \snippet{code_warntype} to automatic type stability checks.

Another notable optimization came from using \snippet{Ptr} objects to pass views of arrays.
Because Julia is a garbage collected language, most objects in Julia must be allocated on the heap,
then garbage collected when they are no longer used.
However, certain objects, such as integers and floating point numbers can be allocated on the stack,
so they don't need garbage collection.
%needs some citations here
So, by passing \snippet{Ptr} objects, which can be stack allocated,
instead of \snippet{SubArray} objects, which must be heap allocated,
the heap allocations and garbage collection can be greatly reduced in some performance critical sections
However, this prevents use of higher level interfaces that Julia normally provides,
such as the array length being stored with the array its self.
This had much less of a performance increase that improving type stability,
however, it is not a standard optimization for Julia so it deserved mention.

Other, common and minor optimizations were also made.
This optimizations include skipping unnecessary bounds checks
and ensuring arrays data was accessed in the order it is stored in memory.
Additionally, function calls in frequently used loops were replaced with specialized, inlined versions
that take into account the context of the calling method
and avoid using generalized methods that produce extra computations.
%REVIEW should I go into more detail on the specific example (getting a row in SpMV)

\section{Comparisons with Other Distributed Libraries}

\subsection{Comparison with Epetra}

Epetra is the base implementation of the Petra object model,
written in a stable subset of C++ \cite{Heroux:2005:Trilinos}.
Because Epetra was used as a template for implementing JuliaPetra,
the APIs for JuliaPetra are similar to those for Epetra.
The similarities between JuliaPetra and Epetra can be seen in how similar the respective implementations
of the power method are.
The differences in features between the implementation languages result in differences
between JuliaPetra's and Epetra's APIs.
For example, JuliaPetra takes advantage of higher level structures
such as type templating and thrown exceptions.

JuliaPetra is able to achieve the same performance as Epetra on large instances of the power method,
as shown in Table~\ref{tab:timing-results}.
Since Julia uses just in time compiling, the JuliaPetra power method has extra start up costs compared to
the Epetra version. However, since each specialized method needs to be compiled only once,
this is a fixed cost during the first evaluation.
Additionally, JuliaPetra uses runtime dispatch in a few locations, such as with
\snippet{Comm} objects, which adds additional overhead compared to Epetra.

\subsection{Comparison with DistributedArrays.jl}

DistributedArrays.jl is a similar library to JuliaPetra that supports
distributed arrays in parallel environments \cite{Github:DA}.
It uses Julia's built in fork-and-join model for parallelism
\cite{Bezanson:2017:FreshApproach}.
DistributedArrays.jl offers a number of advantages over JuliaPetra, however, it is not able to
match the speed of JuliaPetra.

DistributedArrays.jl has advantages over JuliaPetra because it is better designed to function with
existing Julia code.
For example, it is built around the standard Julia parallelization and array APIs and models.
Another advantage is that DistributedArrays.jl uses an instance of \snippet{AbstractArray}
for a process's local storage, this allows automatic support for different array structures.
Finally, by keeping the program logic on the master process, DistributedArrays.jl can be used to
do distributed computations while using Julia's read-eval-print loop.

However, DistributedArrays.jl is not able to match the speed of JuliaPetra.
JuliaPetra is able to run within half of the execution time of DistributedArrays.jl.
This difference in time is caused by JuliaPetra's less frequent and more efficient inter-process communication,
since it is built on MPI's low level interface as opposed to Julia's high level, function based interface.

\subsection{Timing Results}

\begin{table}
	\begin{tabular}{|c c|r|r|r||r|r|}
		\hline
			\multicolumn{2}{|c|}{Equations Per Process}
			& JuliaPetra
			& Epetra
			& \multicolumn{1}{m{1.8cm}||}{Distributed\-Arrays.jl}
			& \multicolumn{1}{m{1.75cm}|}{JuliaPetra / Epetra}
			& \multicolumn{1}{m{1.8cm}|}{JuliaPetra / Distributed\-Arrays.jl} \\
		\hline
			\multicolumn{7}{|l|}{4 Processors}\\
		\hline
			100,000			&Average & 0.22850 & 0.18045 &  & 1.26633 &  \\
			equations		&Minimum & 0.22768 & 0.13394 &  & 1.69985 &  \\
							&Maximum & 0.22933 & 0.21588 &  & 1.06232 &  \\
		\hline
			1,000,000		&Average & 2.82589 & 2.35860 &  & 1.19812 &  \\
			equations		&Minimum & 2.80843 & 2.34777 &  & 1.19621 &  \\
							&Maximum & 2.83711 & 2.36840 &  & 1.19790 &  \\
		\hline
			10,000,000		&Average & 30.7656 & 24.6991 &  & 1.24562 &  \\
			equations		&Minimum & 30.5471 & 24.5430 &  & 1.24464 &  \\
							&Maximum & 31.2470 & 24.7841 &  & 1.26077 &  \\
		\hline
			\multicolumn{7}{|l|}{16 Processors}\\
		\hline
			100,000			&Average & 0.54401 & 0.55244 &  & 0.98474 &  \\
			equations		&Minimum & 0.52902 & 0.50857 &  & 1.04021 &  \\
							&Maximum & 0.55428 & 0.56908 &  & 0.97400 &  \\
		\hline
			1,000,000		&Average & 7.44313 & 7.43903 &  & 1.00055 &  \\
			equations		&Minimum & 7.41123 & 7.42147 &  & 0.99862 &  \\
							&Maximum & 7.53777 & 7.45056 &  & 1.01171 &  \\
		\hline
			10,000,000		&Average & 7.47297 & 73.9551 &  & 0.99348 &  \\
			equations		&Minimum & 73.1307 & 73.9172 &  & 0.98936 &  \\
							&Maximum & 74.1664 & 73.9953 &  & 1.00231 &  \\
		\hline
				\multicolumn{7}{|l|}{20 Processors}\\
		\hline
			100,000			&Average & 0.76502 & 0.72094 &  & 1.06114 &  \\
			equations		&Minimum & 0.73781 & 0.69893 &  & 1.05563 &  \\
							&Maximum & 0.79881 & 0.76437 &  & 1.04507 &  \\
		\hline
			1,000,000		&Average & 9.67784 & 9.74738 &  & 0.99287 &  \\
			equations		&Minimum & 9.47876 & 9.61908 &  & 0.98541 &  \\
							&Maximum & 9.85454 & 9.86118 &  & 0.99933 &  \\
		\hline
			10,000,000		&Average & 96.6538 & 97.4296 &  & 0.99204 &  \\
			equations		&Minimum & 95.1889 & 96.7890 &  & 0.98347 &  \\
							&Maximum & 97.5618 & 98.0169 &  & 0.99536 &  \\
		\hline
	\end{tabular}
	
	\caption{Timing results of various power method implementations.  All times are in seconds.}
	\label{tab:timing-results}
\end{table}

As can be seen in Table~\ref{tab:timing-results}, JuliaPetra is very close to the performance
of Epetra for large problem sizes.
Additionally, JuliaPetra is able to significantly outperform DistributedArrays.jl, performing in less than
half the time of DistributedArrays.jl for all but the largest problems.

The timings come from implementations of the power method for finding the largest eigenvalue of a matrix
\cite{Gu:2000:PowerMethod}.
This tests the performance of vector dot product, vector L2-norm and sparse matrix-vector multiplication functionalities.
The matrices used were various sizes of symmetric tridiagonal matrices, with the diagonal containing 2's
and the off diagonals containing -1's, that were distributed over different numbers of processors.
Each combination of implementation, problem size and number of processors was run five times
and the average, minimum and maximum are presented.
All implementations were run once before collecting the timings, to ensure all Julia code was
compiled before the timing started.
Additionally, the setup for the problems were not counted as part of the times.

The MPI implementation was MPICH version 3.2 with TCP used for all inter-process communication.
The Julia tests were run with the Julia v1.0.0 precompiled binary for Linux with the \snippet{-O3} flag enabled at run time.
The Epetra tests were compiled with GCC 4.8.5, the \snippet{-O3} flag and Trilinos 12.10.1.
Tests were run on a Red Hat Enterprise Linux Workstation, version 7.3,
with a 20 core Intel E5-2698 v4 CPU.

\section{Conclusion}

JuliaPetra is an implementation of the Petra object model in the Julia programming language.
So, it provides an interface for distributed solver algorithms to interact and build on each other.
Additionally, by performing at the same speeds as Epetra,
it allows Julia to compete with C++ for these types of high performance computations.

%REVIEW are there other things that should be discussed in the conclusion
	%It feels pretty short

\bibliography{bibliography}
\end{document}
