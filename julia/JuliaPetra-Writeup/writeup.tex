 \documentclass{article}

\title{Implementation of Petra in Julia}
\author{Neil Lindquist}

\newcommand{\juliaSnippet}[1]{\texttt{\detokenize{#1}}}

\begin{document}

\maketitle

\section{Introduction}

Julia is a programing language first publicly released in 2012 \cite{WhyWeCreatedJulia}.
Julia is designed so that it
``has the performance of a statically compiled
language while providing interactive dynamic behavior and productivity like Python, LISP or
Ruby'' \cite{JuliaDesignPaper}.
%TODO look into referencing use in HPC

%TODO summary of Petra object model
The Petra Object Model is the design used in Trilinos for objects commonly used in solver algorithms.
\cite{OverviewOfTrilinos}
Petra libraries provide parallel, distributed matrices, vectors and graphs for other packages to use.

\section{Design of JuliaPetra}

%TODO general design notes
The design of JuliaPetra follows closely with that of ePetra and tPetra.


%TODO talk about what is and isn't ideomatic Julia
The JuliaPetra is not really idiomatic Julia code.
	%TODO Data Structures don't mesh with Julia Arrays
		%TODO look at fixing that
The data structures used by JuliaPetra don't currently interact with Julia Arrays.
They inherit from \juliaSnippet{DistObject} as objects that
	%TODO MPI is different from Julia's built in comm model


%TODO talk about how Operator is duck typed

\section{Optimizing JuliaPetra}
%REVIEW Is this best on its own? Or is it better as a subsection of the previous?

Type Stability was one of the most important factors in improving the speed of JuliaPetra.
A function is Type Stability if the type inferencer is able to deduce that a variable will only
hold objects of a specific, concrete type.
%REVIEW figure out if I need a cite on the definition of Type Stability
When the type inferencer is able to infer a variable's type well enough that only a single
method matches for a particular function call, that method call may be inlined. \cite{JuliaDesignPaper}
This removes method dispatch and call overhead at run time.
%TODO might need a citation
%TODO mention that even if inlining doesn't happen the method instance is saved to skip dispatch?
%TODO comment about Julia compiling mutliple versions of a method for different types

Given the importance of Type Stability for high performance Julia code, it is important it be able
to determine if a critical section of code is type stable.
However, the only tool Julia provides for checking type stability is the \juliaSnippet{code_warntype}
function and the related \juliaSnippet{@code_warntype} macro.  This function takes a function and list
of types, then outputs a version of the function for arguments of the given types after some optimizations
have occurred with non-concrete types marked in red.
%REVIEW clean up the previous sentence
This requires manually checking a function for each combination of argument types that should be optimized.
Furthermore, there are not any tools provided by the Julia community to make checking type stability easier.
So, to check a function automatically and for multiple function signatures at a time, I had to write some
utilities to do the checking.
%TODO call it out as the "TypeStability.jl package"?  Should look into publishing it publicly


In the matrix vector multiplication loop, there is a lot of use for getting a view of a chunk of an array.
However, creating an actual array view object requires allocating to the heap, and thus garbage collection.
The work around used in JuliaPetra at various internal locations is to use the \juliaSnippet{Ptr} API.
The \juliaSnippet{Ptr} API is designed for interacting with C++, however it allows creating a \juliaSnippet{Ptr} to location in
an array and dereferencing elements in that array.
As bits types, \juliaSnippet{Ptr} objects and integers are able to be stack allocated.
Additionally, \juliaSnippet{Tuple} objects are optimized so that they can be stack allocated.
Thus, by representing a view of an array as a \juliaSnippet{(Ptr, Integer)} tuple, the view
can be stack allocated, meaning it does not need to be garbage collected.
This improves the speed of the matrix vector multiplication loop.
%TODO figure out what should be cited from where in this paragraph


%TODO talk about need for agressive inlining
Another optimization was use of the \juliaSnippet{@inline} macro, which indicates to the
compiler that the annotated function should be inlined.
This is useful for marking internal functions that are only used a few times, to make sure
that the compiler inlines them, even if it would otherwise not.
%REVIEW does this really need to be a paragraph?  It's kind of a basic thing, force critical functions to be inlined
%TODO check if @inline still even makes a difference


%TODO get data to support optimizations

\section{Comparing with ePetra}

%TODO figure out comparison of readability of code


%TODO mention timings?  comment on julia's startup time?


\section{Comparing with DistributedArrays.jl}

%TODO compare - I made alot of notes in the outline

\section{Results}

%TODO show timing data


%TODO discuss methodology used to get data


\section{Conclusion}

%TODO say JP runs as fast as the C++ version
%TODO mention how not ideomatic Julia, but ideomatic version was real slow


\bibliographystyle{acm}
\bibliography{bibliography}
\end{document}